%
% Copyright 2019 Markus Borg, Lund University
%
% This work is licensed under a Creative Commons Attribution-ShareAlike 4.0 International License.
% See http://creativecommons.org/licenses/by-sa/4.0/
%
% The dodument is based on a LaTeX template developed by Jean-Philippe Eisenbarth
% https://github.com/jpeisenbarth/SRS-Tex
%
\documentclass{scrreprt}
\usepackage{listings}
\usepackage{underscore}
\usepackage[bookmarks=true]{hyperref}
\usepackage[utf8]{inputenc}
\usepackage[english]{babel}
\hypersetup{
    bookmarks=false,    % show bookmarks bar?
    pdftitle={Lab 0},    % title
    pdfauthor={Markus Borg},                     % author
    pdfsubject={TeX and LaTeX},                        % subject of the document
    pdfkeywords={TeX, LaTeX, graphics, images}, % list of keywords
    colorlinks=true,       % false: boxed links; true: colored links
    linkcolor=blue,       % color of internal links
    citecolor=black,       % color of links to bibliography
    filecolor=black,        % color of file links
    urlcolor=purple,        % color of external links
    linktoc=page            % only page is linked
}%
\def\myversion{0.1 }
\date{}
%\title
\usepackage{hyperref}
\begin{document}

\begin{flushright}
    \rule{16cm}{5pt}\vskip1cm
    \begin{bfseries}
    	\LARGE{ETSA02-ADM-LAB0}\\
    	\vspace{1.5cm}
        \Huge{Lab 0}\\
        \vspace{0.5cm}
        Basic development skills\\
        \vspace{0.5cm}
        required for ETSA02\\
        \vspace{1.5cm}
        \LARGE{Version \myversion}\\
        %\LARGE{Version \myversion approved}\\
        \vspace{1.5cm}
        Prepared by Markus Borg\\
        %\vspace{1.5cm}
        Dept. of Computer Science, Lund University\\
        \vspace{1.5cm}
        \today\\
    \end{bfseries}
\end{flushright}

%\tableofcontents

\chapter*{Revision History}

\begin{center}
    \begin{tabular}{|c|c|c|c|}
        \hline
	    Name & Date & Reason For Changes & Version\\
        \hline
	    Markus Borg & 2019-02-02 & Initial skeleton. & 0.1\\
        \hline
    \end{tabular}
\end{center}

\chapter{Introduction}
Lab 0 will introduce some core concepts that you will work with extensively throughout this course. This introductory lab has been added to ensure that everyone gets a chance to practice some fundamental development skills -- the students following the course often represent a wide range of backgrounds. Some of you will wonder why it covers elementary concepts, but from experience we know that the content in this pre-lab is not obvious to everyone. You may certainly skip all exercises related to topics you already know well, but be prepared to answer some questions about Lab 0 during the Lab 1 session.

Lab 0 covers the following topics:
\begin{itemize}
 \item Fundamentals of file systems
 \item Command line interfaces
 \item Version control software
 \item Test automation
\end{itemize} 

\chapter{Fundamentals of file systems}

\chapter{Command line interfaces}

\chapter{Version control software}

\chapter{Test automation}

\end{document}